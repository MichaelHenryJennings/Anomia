\documentclass[12pt,letterpaper]{article}
\usepackage{fullpage}
\usepackage[top=2cm, bottom=4.5cm, left=2.5cm, right=2.5cm]{geometry}
\usepackage{amsmath,amsthm,amsfonts,amssymb,amscd}
\usepackage{lastpage}
\usepackage{enumerate}
\usepackage{fancyhdr}
\usepackage{mathrsfs}
\usepackage{xcolor}
\usepackage{graphicx}
\usepackage{listings}
\usepackage{hyperref}
\usepackage{seqsplit}

\renewcommand{\labelenumi}{(\alph{enumi})}
\hypersetup{%
  colorlinks=true,
  linkcolor=blue,
  linkbordercolor={0 0 1}
}
 
\renewcommand\lstlistingname{Algorithm}
\renewcommand\lstlistlistingname{Algorithms}
\def\lstlistingautorefname{Alg.}
\makeatletter
\newcommand*{\rom}[1]{\expandafter\@slowromancap\romannumeral #1@}
\makeatother
\lstdefinestyle{Python}{
    language        = Python,
    frame           = lines, 
    basicstyle      = \footnotesize,
    keywordstyle    = \color{blue},
    stringstyle     = \color{green},
    commentstyle    = \color{red}\ttfamily
}

\setlength{\parindent}{0.0in}
\setlength{\parskip}{0.05in}


\pagestyle{fancyplain}
\headheight 35pt
\lhead{Michael Jennings}        
\chead{\textbf{\Large Anomia}}
\rhead{}
\lfoot{}
\cfoot{\small\thepage}
\rfoot{}
\headsep 1.5em

\newtheorem*{claim}{Claim}
\newtheorem*{lemma}{Lemma}

\begin{document}

\section[0]{Introduction}

\subsection[0]{Overview}
Anomia is a keyword-free programming language.

\subsection[1]{About}
Authors: Michael Jennings. \newline
Started: June 5th, 2024. \newline
GitHub: https://github.com/MichaelHenryJennings/Anomia \newline
Current Version: v0.0.1 (for complete version history, see GitHub repository)

\section[1]{Specifications}

% Miscellaneous Notes:

% \begin{enumerate}
%   \item Variable declarations and assignments 
%   \item Special functions: \begin{enumerate}
%   \item hi
%   \end{enumerate}
% \end{enumerate}

\subsection[0]{Grammar}

\subsubsection[0]{Formal}
\subsubsection[1]{Informal}
Anomia programs comprise a block of statements, which are either expressions (usually) followed by a semicolon or blocks of statements. Blocks are delimited by curly braces (``\{'' and ``\}'') with zero or more statements inside. Each expression has one of the following types:
\begin{enumerate}[]
  \item \textbf{Variable declaration:} A variable name, then a colon (``:''), then a type name or primitive type identifier.
  \item \textbf{Type declaration:} A type name, then a double colon (``::''), then a class declaration, type name, or primitive type identifier.
  \item \textbf{Assignment:} A variable name, then an equals sign (``=''), then a value.
  \item \textbf{If(else):} An expression (the ``condition''), then an arrow (``-\textgreater''), then a statement to execute if the condition was nonzero (true), then (optionally) a tilde ("~") followed by another statement to execute if the condition was zero (false).
  \item \textbf{Loop:} An expression (the "condition"), then a double-sided arrow (``\textless-\textgreater''), then a statement to execute while the condition is nonzero (true).
  \item \textbf{Function call:} A function or function name, then a pattern of arguments.
  \item \textbf{Print:} The dollar symbol (``\$''), then a pattern of arguments, the first of which is a format string; behaves exactly like a function call.
  \item \textbf{Size:} The pound symbol (``\#''), then a type name; returns the size in bytes of that type.
  \item \textbf{Reference:} A backtick, then a name or block of statements (used for function declarations).
  \item \textbf{Unary Operation:} A unary operator (``-'', ``!'', or ``@''), then an expression.
  \item \textbf{Binary Operation:} An expression, then a binary operator, then another expression.
  \item \textbf{Pattern:} A set of comma-separated items (of various types) enclosed in square brackets (``['' and ``]'').
  \item \textbf{Value:} A number or other literal value (e.g. character literal).
  \item \textbf{Name:} A single string of letters (capital or lowercase) and underscores.
\end{enumerate}

\section[2]{Appendices}

\subsection[0]{Version History}
v0.0.0 (6/5/2024): initial commit \newline
v0.0.1 (6/5/2024): simple high-level specifications drafted

\end{document}